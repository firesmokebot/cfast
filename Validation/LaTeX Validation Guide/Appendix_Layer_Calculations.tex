\chapter{Layer Height and the Average Upper and Lower Layer Temperatures}
\label{Appendix_layerheight}

Fire protection engineers often need to estimate the location of the
interface between the hot, smoke-laden upper layer and the cooler
lower layer in a burning compartment.  Relatively simple fire models,
often referred to as {\em two-zone models}, compute this quantity
directly, along with the average temperature of the upper and lower
layers.  In a computational fluid dynamics (CFD) model like FDS, there
are not two distinct zones, but rather a continuous profile of
temperature. Nevertheless, there are methods that have been developed
to estimate layer height and average temperatures from a continuous
vertical profile of temperature. One such
method~\cite{Janssens:1992} is as follows: Consider a continuous
function $T(z)$ defining temperature $T$ as a function of height above
the floor $z$, where $z=0$ is the floor and $z=H$ is the
ceiling. Define $T_u$ as the upper layer temperature, $T_l$ as the
lower layer temperature, and $z_{int}$ as the interface
height. Compute the quantities:
\begin{eqnarray*} (H-z_{int})\; T_u + z_{int} \; T_l = \int_0^H \; T(z) \; dz &=& I_1 \\
                  (H-z_{int})\; \frac{1}{T_u} + z_{int} \; \frac{1}{T_l} = \int_0^H \; \frac{1}{T(z)} \; dz &=& I_2 \end{eqnarray*}
Solve for $z_{int}$:
\be z_{int} = \frac{ T_l(I_1 \, I_2 - H^2)}{I_1+I_2 \, T_l^2 - 2\, T_l \, H} \ee
Let $T_l$ be the temperature in the lowest mesh cell and, using
Simpson's Rule, perform the numerical integration of $I_1$ and
$I_2$. $T_u$ is defined as the average upper layer temperature via
\be (H-z_{int})\; T_u = \int_{z_{int}}^H \; T(z) \; dz \ee
Further discussion of similar procedures can be found in Ref.~\cite{He:1998}.



\clearpage