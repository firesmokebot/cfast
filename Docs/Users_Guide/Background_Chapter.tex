
\chapter{Background}

CFAST is a two-zone fire model used to calculate the evolving distribution of smoke, fire gases and temperature throughout compartments of a building during a fire. These can range from very small containment vessels, on the order of 1 m3 to large spaces on the order of 1000 m3.  This guide describes how to obtain the model, verify its correct installation, create input data in an appropriate form, and analyze of the output of a simulation. 

The modeling equations used in CFAST take the mathematical form of an initial value problem for a system of ordinary differential equations (ODEs).  These equations are derived using the conservation of mass, the conservation of energy (equivalently the first law of thermodynamics), the ideal gas law and relations for density and internal energy.  These equations predict as functions of time quantities such as pressure, layer height and temperatures given the accumulation of mass and enthalpy in the two layers.  The CFAST model then consists of a set of ODEs to compute the environment in each compartment and a collection of algorithms to compute the mass and enthalpy source terms required by the ODEs.  The formulation of the equations, their solution, and discussion of validation and verification of the code are presented in a companion document \cite{CFAST_Tech_Guide_6}.

All of the data to run the model is contained in a primary data file, together with databases for objects, thermophysical properties of boundaries, and sample prescribed fire descriptions \cite{Gross:1985}. These files contain information about the building geometry (compartment sizes, materials of construction, and material properties), connections between compartments (horizontal flow openings such as doors, windows, vertical flow openings in floors and ceilings, and mechanical ventilation connections), fire properties (fire size and species production rates as a function of time), and specifications for detectors, sprinklers, and targets (position, size, heat transfer characteristics, and flow characteristics for sprinklers). Materials are defined by their thermal conductivity, specific heat, density, thickness, and burning behavior. Throughout the discussion on the model inputs, notes are included to provide additional insight on the model�s operation.

The outputs of CFAST are the sensible variables that are needed for assessing the environment in a building subjected to a fire. These include temperatures of the upper and lower gas layers within each compartment, the ceiling/wall/floor temperatures within each compartment, the visible smoke and gas species concentrations within each layer, target temperatures and sprinkler activation time. 

Many of the outputs from the CFAST model are relatively insensitive to uncertainty in the inputs for a broad range of scenarios. However, the more precisely the scenario is defined, the more accurate the results will be. Not surprisingly, the heat release rate is the most important variable, because it provides the driving force for fire-driven flows. Other variables related to compartment geometry such as compartment height or vent sizes, while important for the model results, are typically more easily defined for specific design scenarios than fire related inputs. 

The first public release of CFAST was version 1.0 in June of 1990. This version was restructured from FAST \cite{Models:FAST} to incorporate the "lessons learned" from the zone model CCFM \cite{Models:CCFM}, namely that modifications and additions to the model are easier and more robust if the components such as the physical routines are separated from the solver code used by the model. Version 2 was released as a component of Hazard 1.2 in 1994 \cite{Models:HAZARDI}. The first of the 3.x series was released in 1995 and included a vertical flame spread algorithm, ceiling jets and non-uniform heat loss to the ceiling, spot targets, and heating and burning of multiple objects in addition to multiple prescribed fires. Ignition was assigned based on a critical heat flux, a critical temperature, or a critical time input by the user. As CFAST evolved over the next five years, version 3 included smoke and heat detectors, suppression through heat release reduction, better characterization of flow through doors and windows, vertical heat conduction through ceiling/floor boundaries, and non-rectangular compartments. In 2000, version 4 was released and included horizontal heat conduction through walls, and horizontal smoke flow in corridors. Version 5 improved the combustion chemistry.  Version 6 included a new user interface written for Windows and revisions to the input file and model. The current version is 6.3
