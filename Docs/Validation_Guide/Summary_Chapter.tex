	\chapter{Summary and Conclusions}

How to best quantify the comparisons between model predictions and experiments is not obvious. The necessary and perceived level of agreement for any variable is dependent upon both the typical use of the variable in a given simulation, the nature of the experiment, and the context of the comparison in relation to other comparisons being made. For instance, the user may be interested in the time it takes to reach a certain temperature in the room, but have little or no interest in peak temperature for experiments that quickly reach a steady-state value. Insufficient experimental data and understanding of how to compare the numerous variables in a complex fire model prevent a complete validation of the model. 

A true validation of a model would involve proper statistical treatment of all the inputs and outputs of the model with appropriate experimental data to allow comparisons over the full range of the model. Thus, the comparisons of the differences between model predictions and experimental data discussed here are intentionally simple and vary from test to test and from variable to variable due to the changing nature of the tests and typical use of different variables.

Table \ref{tab:Summary_Relative_Diffs} summarizes the comparisons in this report.

\begin{table}

\caption{Summary of Model Comparisons}
\label{tab:Summary_Relative_Diffs}
\vspace{0.1in}

\IfFileExists{FIGURES/ScatterPlots/validation_statistics.tex}{\begin{center}
\begin{longtable}{|c|c|c|c|c|c|}
\hline
Quantity & Number of & Number of & $2\widetilde{\sigma}_E$ & $2\widetilde{\sigma}_M$ & Bias \\
         & Datasets  & Points    &                         &                         &      \\ \hline \hline
\endfirsthead
\hline
Quantity & Number of & Number of & $2\widetilde{\sigma}_E$ & $2\widetilde{\sigma}_M$ & Bias \\
         & Datasets  & Points    &                         &                         &      \\ \hline \hline
\endhead
HGL Temperature & 11 & 219 & 0.14 & 0.48 & 1.14 \\ \hline
HGL Temperature: Forced Ventilation & 7 & 91 & 0.14 & 0.39 & 1.25 \\ \hline
HGL Temperature: Natural Ventilation & 8 & 104 & 0.14 & 0.51 & 1.11 \\ \hline
HGL Temperature: No Ventilation & 3 & 22 & 0.14 & 0.69 & 1.32 \\ \hline
HGL Depth & 7 & 53 & 0.13 & 0.45 & 0.98 \\ \hline
Ceiling Jet Temperature & 6 & 208 & 0.10 & 0.44 & 1.23 \\ \hline
Plume Temperature & 4 & 51 & 0.14 & 0.42 & 1.17 \\ \hline
Oxygen Concentration & 6 & 40 & 0.09 & 0.61 & 1.04 \\ \hline
Carbon Dioxide Concentration & 5 & 31 & 0.09 & 0.49 & 0.85 \\ \hline
Smoke Concentration & 1 & 15 & 0.33 & 1.51 & 3.78 \\ \hline
Compartment Over-Pressure & 1 & 9 & 0.40 & 0.84 & 1.30 \\ \hline
Open Compartment Over-Pressure & 2 & 8 & 0.40 & 0.79 & 1.47 \\ \hline
Target Heat Flux & 1 & 100 & 0.20 & 1.30 & 1.01 \\ \hline
Wall Heat Flux & 6 & 121 & 0.20 & 0.75 & 1.21 \\ \hline
Target Temperature & 2 & 73 & 0.14 & 1.37 & 1.44 \\ \hline
Wall Temperature & 5 & 122 & 0.14 & 0.94 & 1.10 \\ \hline
Smoke Alarm Activations & 7 & 125 & 0.33 & 0.98 & 1.05 \\ \hline
Sprinkler Activation Time & 2 & 68 & 0.20 & 0.52 & 0.84 \\ \hline
\end{longtable}
\end{center}
}{\typeout{Error: Missing file FIGURES/ScatterPlots/validation_statistics.tex}}

\end{table}

Four of the quantities were seen to require additional care when using the model to evaluate the given quantity.  This typically indicates limitations in the use of the model.  A few notes on the comparisons are appropriate:

\begin{itemize}
\item CFAST typically predicts plume temperature near to experimental uncertainty, but tends to under-predict temperatures nearer to the fire source and over-predict temperatures farther away.
\item CFAST typically over-predicts smoke concentration.  Predicted concentrations for open-door tests are within experimental uncertainties, but those for closed-door tests are far higher.
\item With exceptions, CFAST predicts cable surface temperatures within experimental uncertainties.  Total heat flux to targets is typically predicted to within about 30~\%, and often under-predicted.  Radiative heat flux to targets is typically over-predicted compared to experimental measurements, with higher relative difference values for closed-door tests.  Care should be taken in predicting localized conditions (such as target temperature and heat flux) because of inherent limitations in all zone fire models.
\item Predictions of compartment surface temperature and heat flux are typically within 10~\% to 30~\%.  Generally, CFAST over-predicts the far-field fluxes and temperatures and under-predicts the near-field measurements.  This is consistent with the single representative layer temperature assumed by zone fire models.
\end{itemize}

CFAST predictions in this validation study were consistent with numerous earlier studies, which show that the use of the model is appropriate in a range of fire scenarios.  The CFAST model has been subjected to extensive evaluation studies by NIST and others.  Although differences between the model and the experiments were evident in these studies, most differences can be explained by limitations of the model as well as of the experiments.  Like all predictive models, the best predictions come with a clear understanding of the limitations of the model and the inputs provided to perform the calculations.