
\chapter{Overview}

CFAST is a fire model capable of predicting the fire-induced environmental conditions as a function of time for single- or multi-compartment scenarios.  Toward that end, the CFAST software calculates the evolution of temperature, smoke, and fire gases throughout a building during a user-prescribed fire.  The model was developed, and is maintained, by the Fire Research Division of the National Institute of Standards and Technology (NIST).

CFAST is a two-zone model, in that it subdivides each compartment into two zones, or control volumes and the two volumes are assumed to be homogeneous within each zone.  This two-zone approach has evolved from observations of layering in actual fires and real-scale fire experiments.  Differential equations that approximate solution of the mass and energy balances of each zone, equations for heat conduction into the walls, and the ideal gas law simulate the environmental conditions generated by a fire.

This document describes the underlying structure of the CFAST model and the processes used during the development and deployment of the model. It is intended to provide guidelines for

\begin{itemize}
\item the planning for modifications to the model,
\item any required reviews for both software and associated documentation of the model,
\item testing to be conducted prior to the release of an updated model,
\item problem reporting and resolution,
\item retention of records, source code, and released software over the life of the code.
\end{itemize}

\section{What is Model Validation?}

Key to ensuring the quality of the software are ongoing validation testing of the model. Validation typically involves comparing model simulations with experimental measurements. To say that CFAST is ``validated'' means that the model has been shown to be of a given level of accuracy for a given range of parameters for a given type of fire scenario. Although the CFAST developers periodically perform validation studies, it is ultimately the end user of the model who decides if the model is adequate for the job at hand. Thus, this Guide does not and cannot be considered comprehensive for every possible modeling scenario.

Although there are various definitions of model validation, for example those contained in
ASTM E 1355~\cite{ASTM:E1355}, most define it as the process of determining how well the mathematical model
predicts the actual physical phenomena of interest.
Validation typically involves (1) comparing model predictions with experimental measurements, (2) quantifying the differences in light of uncertainties in
both the measurements and the model inputs, and (3) deciding if the model is appropriate for the given application. This Guide only does (1) and (2). Number (3) is
the responsibility of the model user.

A common question asked of any mathematical model is whether it is validated. To say that CFAST is
``validated'' means that the model has been shown to be of a given level of accuracy for a given range of parameters for a given
type of fire scenario. Although the CFAST developers continuously perform validation studies, it is ultimately the end user of the model who
decides if the model is adequate for the job at hand. Thus, this Guide provides the raw material for a validation study, but it does not
and cannot be considered comprehensive.

The following sections discuss key issues that you must consider when deciding whether or not CFAST has been validated. It depends on (a) the scenarios
of interest, (b) the predicted quantities, and (c) the desired level of accuracy. Keep in mind that CFAST can be used to model most any compartment fire scenario and predict 
quantities of interest, but the prediction may not be accurate because of limitations in the description of the fire physics, and also because of limited
information about the fuels, geometry, and so on.


\section{How to Use this Guide}

 This guide presents a compilation of past and present validation exercises for the CFAST model are presented.  As CFAST continues to develop, it will expand to include new experimental
measurements of newly modeled physical phenomena. With each change in,
the plots and graphs are all be redone to ensure that changes to the model are consistent with any changes in the predictions and the overall accuracy of the model. If you are embarking on a validation study, you might want to consider the following steps:
\begin{enumerate}
\item Survey Chapter~\ref{Survey_Chapter} to learn about past efforts by others to validate the model for
applications similar to yours. Keep in mind that most of the referenced validation exercises have been performed with
older versions of CFAST, and you may want to obtain the experimental data and the old FDS input files and redo the
simulations with the version of CFAST that you plan to use.
\item Identify in Chapter~\ref{Experiment_Chapter} experimental data sets appropriate for your application. In particular,
the summary of the experiments found in table~\ref{Test_Parameters} contains a table listing various non-dimensional
quantities that characterize the parameters of the experiments. For example, the equivalence ratio of a compartment fire
experiment indicates the degree to which the fire was over or under-ventilated.
To say that the results of a given experiment are relevant to
your scenario, you need to demonstrate that its parameters ``fit'' within the parameter space outlined in
Table~\ref{Test_Parameters}.
\item Search the Table of Contents to find comparisons of CFAST simulations with the relevant experiments.
For a given experiment, there may be numerous measurements of quantities like the gas temperature,
heat flux, and so on.
It is a challenge to sort out all the plots and graphs of all the different quantities and come to some general conclusion.
For this reason, this Guide is organized by output quantity, not by individual experiment or fire scenario.
In this way, it is possible to assess, over a range of
different experiments and scenarios, the performance of the model in predicting a given quantity.
Overall trends and biases become much more clear when the data is organized this way.
\end{enumerate}
The experimental data sets and CFAST input files described in this Guide are all managed via the on-line project archiving system.
You might want to re-run examples of interest
to better understand how the calculations were designed, and how changes in the various parameters might affect the results.
This is known as a {\em sensitivity study},
and it is difficult to document all the parameter variations of the calculations described in this report.
Thus, it is a good idea to determine which of the input parameters are particularly important.

\section{Blind, Specified, and Open Validation Experiments}

ASTM E 1355~\cite{ASTM:E1355} describes three basic types of validation calculations -- {\em Blind}, {\em Specified}, and
{\em Open}.
\begin{description}
\item [Blind Calculation:] The model user is provided with
a basic description of the scenario to be modeled. For this
application, the problem description is not exact; the model
user is responsible for developing appropriate model inputs
from the problem description, including additional details of
the geometry, material properties, and fire description, as
appropriate. Additional details necessary to simulate the scenario
with a specific model are left to the judgement of the
model user. In addition to illustrating the comparability of
models in actual end-use conditions, this will test the ability of
those who use the model to develop appropriate input data for
the models.
\item [Specified Calculation:] The model user is provided
with a complete detailed description of model inputs, including
geometry, material properties, and fire description. As a
follow-on to the blind calculation, this test provides a more
careful comparison of the underlying physics in the models
with a more completely specified scenario.
\item [Open Calculation:] The model user is provided with
the most complete information about the scenario, including
geometry, material properties, fire description, and the results
of experimental tests or benchmark model runs which were
used in the evaluation of the blind or specified calculations of
the scenario. Deficiencies in available input (used for the blind
calculation) should become most apparent with comparison of
the open and blind calculation.
\end{description}
The calculations presented in this Guide all fall into the {\em Open} category. There are several reasons for this, the first
being the most practical:
\begin{itemize}
\item All of the calculations presented in this Guide are routinely rerun. The
fact that the experiments have already been performed and the results are known automatically qualify these
calculations as {\em Open}.
\item Some of the calculations described in this Guide did originally fall into the {\em Specified} category because they
were first performed before the experiments were conducted. However, in almost every case, the experiment was not conducted
exactly as specified, and the calculation results were not particulary useful in determining the accuracy of the model.
\item None of the calculations were truly {\em Blind}, even those performed prior to the experiments. The purpose of a
{\em Blind} calculation is to assess the degree to which the choice of input parameters affects the outcome. However,
in such cases it is impossible to discern the uncertainty associated from the choice of input parameters from that associated
with the model itself. The primary purpose of this Guide is to quantify the uncertainty of the model itself, in which
case {\em Blind} calculations are of little value.
\end{itemize}


\section{Software Development and Quality Assurance}

The development and maintenance of CFAST is guided by software quality assurance measures for the planning of modifications to the model that provide required reviews for both software and associated documentation of the model, define testing to be conducted prior to the release of an updated model, describe problem reporting and resolution procedures, and ensure all records, source code, and released software is kept available for the life of the code.  

The internal structure of the model also has an impact on the ease of modification and correctness of the model.  The method for incorporating new phenomena and ensuring the correctness of the code was adopted as part of the consolidation of CCFM and FAST and has resulted in a more transparent, transportable and verifiable numerical model. This transparency is crucial to a verifiable and robust numerical implementation of the predictive model as discussed in the sections on code checking and numerical analysis.  More recently, all of the software development and software tracking has been made available on the web to further enhance the software development process.

\section{Model Validation Scenarios}

Key to ensuring the correctness and accuracy of the model are comparisons of the model with both earlier versions of the model and with documented experimental data applicable to the intended range of application of the model. When doing a validation study, the first question to ask is, ``What is the application?'' There are countless fire scenarios to consider, but from the point of view of validation, it is useful to divide them into two classes -- those for which the fire is {\em specified} as an input to the model and those for which the fire must be estimated by use of the model. The former is often the case for a design application, the latter for a forensic reconstruction. Consider each in turn.

Design  applications  typically  involve  an existing  building  or  a building  under  design. A  so-called  ``design  fire'' is  prescribed either by  a regulatory authority  or by the engineers  performing the analysis. Because the  fire's heat release rate is  specified, the role of the model is to predict  the transport of heat and combustion products throughout  the room or  rooms of  interest. Ventilation  equipment is often included  in the simulation, like fans,  blowers, exhaust hoods, ductwork,  smoke management systems,  {\em etc.} Sprinkler  and heat and smoke detector activation are also of interest.  The effect of the sprinkler spray on the fire is usually less of interest since the fire is  prescribed rather  than  predicted. Detailed  descriptions of  the contents of the building are usually not necessary because these items are not assumed to be burning,  and even if they are, the burning rate will be  fixed, not predicted.  Sometimes, it is necessary  to predict the heat  flux from the fire  to a nearby ``target,''  and even though the target  may heat up  to some prescribed ignition  temperature, the subsequent spread  of the  fire usually goes  beyond the scope  of the analysis because of the uncertainty  inherent in object to object fire spread.

Forensic reconstructions require the  model to simulate an actual fire based on  information that is collected  after the event,  such as eye witness accounts, unburned materials,  burn signatures, {\em etc.} The purpose  of  the simulation  is  to  connect  a sequence  of  discrete observations  with  a continuous  description  of  the fire  dynamics. Usually,  reconstructions  involve  more gas/solid  phase  interaction because  virtually  all  objects  in  a  given  room  are  potentially ignitable, especially when flashover  occurs. Thus, there is much more emphasis on  such phenomena as  heat transfer to  surfaces, pyrolysis, flame  spread, and suppression.  In general,  forensic reconstructions are more challenging simulations  to perform because they require more detailed  information  about the  room  geometry and contents,  and  there is  much greater uncertainty in the total heat release rate as the fire spreads from object to object.

CFAST has been applied for both design and reconstruction scenarios. For the former, specified design fires are typically used (e.g., reference \cite{NRCNUREG6850}).  For the latter, iterative simulation with multiple model runs allow the user to develop fire growth inputs consistent with observed post fire conditions.

\section{Input Data Required to Run the Model}

All of the data required to run the CFAST model reside in a primary data file, which the user creates.  Some instances may require databases of information on objects, thermophysical properties of boundaries, and sample prescribed fire descriptions.  In general, the data files contain the following information:

\begin{itemize}
\item compartment dimensions (height, width, length)
\item construction materials of the compartment (e.g., concrete, gypsum)
\item material properties (e.g., thermal conductivity, specific heat, density, thickness, heat of combustion)
\item dimensions and positions of horizontal and vertical flow openings such as doors, windows, and vents
\item mechanical ventilation specifications 
\item fire properties (e.g., heat release rate, lower oxygen limit, and species production rates as a function of time) 
\item sprinkler and detector specifications 
\item positions, sizes, and characteristics of targets
\end{itemize}

The CFAST User's Guide \cite{CFAST_Users_Guide_6} provides a complete description of the required input parameters.  Some of these parameters have default values included in the model, which are intended to be representative for a range of fire scenarios.  Unless explicitly noted, default values were used for parameters not specifically included in this validation study.

\section{Property Data}

Required inputs for CFAST must be provided with a number of material properties related to compartment bounding surfaces, objects (called targets) placed in compartments for calculation of object surface temperature and heat flux to the objects, or fire sources.  For compartment surfaces and targets, CFAST needs the density, thermal conductivity, specific heat, and emissivity.

For fire sources, CFAST needs to know the pyrolysis rate of fuel, the heat of combustion, stochiometric fuel-oxygen ratio, yields of important combustion products in a simplified combustion reaction (carbon monoxide, carbon dioxide, soot, and others), and the fraction of energy released in the form of thermal radiation.

These properties are commonly available in fire protection engineering and materials handbooks. Experimentally determined property data may also be available for certain scenarios.  However, depending on the application, properties for specific materials may not be readily available.  A small file distributed with the CFAST software contains a database with thermal properties of common materials.  These data are given as examples, and users should verify the accuracy and appropriateness of the data.

\section{Model Outputs}

Once the simulation is complete, CFAST produces an output file containing all of the solution variables.  Typical outputs include (but are not limited to) the following:

\begin{itemize}
\item environmental conditions in the room (such as hot gas layer temperature; oxygen and smoke concentration; and ceiling, wall, and floor temperatures)
\item heat transfer-related outputs to walls and targets (such as incident convective, radiated, and total heat fluxes)
\item fire intensity and flame height
\item flow velocities through vents and openings
\item detector and sprinkler activation times
\end{itemize}


Thus, for a given fire scenario, there are a number of different quantities that the model predicts A typical
fire experiment can produce hundreds of time histories of point measurements, each of which can be reproduced by the model to some level of accuracy. It is a challenge to sort out all the plots and graphs of all the different quantities and come to some general conclusion. For this reason, this Guide
is organized by output quantity, not by individual experiment or fire scenario. In this way, it is possible to assess, over a range of different experiments and scenarios, the performance of the model in predicting a given quantity. Overall trends and biases become much more clear when the data is organized this way.


\section{Model Accuracy}

The degree of  accuracy for each output variable  required by the user is  highly  dependent on  the  technical  issues  associated with  the analysis.  The user  must ask: How accurate does  the analysis have to be  to  answer  the  technical  question posed?  Thus,  a  generalized definition of the  accuracy required for each quantity  with no regard as  to the specifics  of a  particular analysis  is not  practical and would be limited in its usefulness.

Returning   to    the   earlier   definitions    of   ``design''   and ``reconstruction,'' fire scenarios, design applications  typically are  more accurate because the heat release rate is prescribed rather than predicted, and the    initial    and    boundary    conditions   are    far    better characterized. Mathematically, a design calculation is an example of a ``well-posed''  problem  in  which   the  solution  of  the  governing equations is  advanced in  time starting from  a known set  of initial conditions and constrained by a known set of boundary conditions.  The accuracy of the results is a function of the fidelity of the numerical solution, which is  largely dependent on the quality of the model inputs. This CFAST validation guide includes efforts to date involving well-characterized geometries and prescribed fires. These studies show that  CFAST predictions vary from being within experimental   uncertainty  to  being   about  30~\%   different  than measurements of temperature, heat flux, gas concentration, {\em etc} (see, for example, reference \cite{NRCNUREG1824}).

A reconstruction is an example of an ``ill-posed'' problem because the outcome  is known  whereas  the initial  and  boundary conditions  are not. There is  no single, unique solution to the  problem. Rather, it is possible to simulate numerous fires that produce the given outcome. There is no right or wrong answer, but rather a small set of plausible fire scenarios that are  consistent with the collected evidence and physical laws incorporated into the model. These simulations are then used to demonstrate why the fire behaved as it did  based on the current understanding of fire physics  incorporated in  the model.  Most  often, the  result of  the
analysis is only  qualitative. If there is any  quantification at all, it could be in the time to reach critical events, like a roof collapse or room flashover.

\section{Uses and Limitations of the Model}
CFAST has been developed for use in solving practical fire problems in fire protection engineering.  It is intended for use in system modeling of building and building components.  A priori prediction of flame spread or fire growth on objects is not modeled. Rather, the consequences of a specified fire is estimated. It is not intended for detailed study of flow within a compartment, such as is needed for smoke detector siting.  It includes the activation of sprinklers and fire suppression by water droplets.

The most extensive use of the model is in fire and smoke spread in complex buildings.  The efficiency and computational speed are inherent in the few computation cells needed for a zone model implementation.  The use is for design and reconstruction of time-lines for fire and smoke spread in residential, commercial, and industrial fire applications.  Some applications of the model have been for design of smoke control systems.

\begin{itemize}
\item Compartments:  CFAST is generally limited to situations where the compartment volumes are strongly stratified.  However, in order to facilitate the use of the model for preliminary estimates when a more sophisticated calculation is ultimately needed, there are algorithms for corridor flow, smoke detector activation, and detailed heat conduction through solid boundaries.  This model does provide for non-rectangular compartments, although the application is intended to be limited to relatively simple spaces.  There is no intent to include complex geometries where a complex flow field is a driving force.  For these applications, computational fluid dynamics (CFD) models are appropriate.

\item Gas Layers:  There are also limitations inherent in the assumption of stratification of the gas layers.  The zone model concept, by definition, implies a sharp boundary between the upper and lower layers, whereas in reality, the transition is typically over about 10~\% of the height of the compartment and can be larger in weakly stratified flow.  For example, a burning cigarette in a normal room is not within the purview of a zone model.  While it is possible to make predictions within 5~\% of the actual temperatures of the gas layers, this is not the optimum use of the model.  It is more properly used to make estimates of fire spread (not flame spread), smoke detection and contamination, and life safety calculations.

\item Heat Release Rate:  CFAST does not predict fire growth on burning objects. Heat release rate is specified by the user for one or more fire objects. The model does include the ability to limit the specified burning based on available oxygen. There are also limitations inherent in the assumptions used in application of the empirical models.  As a general guideline, the heat release should not exceed about 1 MW/m$^3$.  This is a limitation on the numerical routines attributable to the coupling between gas flow and heat transfer through boundaries (conduction, convection, and radiation).  The inherent two-layer assumption is likely to break down well before this limit is reached.

\item Radiation:  Because the model includes a sophisticated radiation model and ventilation algorithms, it has further use for studying building contamination through the ventilation system, as well as the stack effect and the effect of wind on air circulation in buildings.  Radiation from fires is modeled with a simple point source approximation.  This limits the accuracy of the model near fire sources. Calculation of radiative exchange between compartments is not modeled.

\item Ventilation and Leakage:  In a single compartment, the ratio of the area of vents connecting one compartment to another to the volume of the compartment should not exceed roughly 1/2 m.  This is a limitation on the plug flow assumption for vents.  An important limitation arises from the uncertainty in the scenario specification.  For example, leakage in buildings is significant, and this affects flow calculations especially when wind is present and for tall buildings.  These effects can overwhelm limitations on accuracy of the implementation of the model.  The overall accuracy of the model is closely tied to the specificity, care, and completeness with which the data are provided.

\item Thermal Properties:  The accuracy of the model predictions is limited by how well the user can specify the thermophysical properties.  For example, the fraction of fuel which ends up as soot has an important effect on the radiation absorption of the gas layer and, therefore, the relative convective versus radiative heating of the layers and walls, which in turn affects the buoyancy and flow.  There is a higher level of uncertainty of the predictions if the properties of real materials and real fuels are unknown or difficult to obtain, or the physical processes of combustion, radiation, and heat transfer are more complicated than their mathematical representations in CFAST.
\end{itemize}

In addition, there are specific limitations and assumptions made in the development of the algorithms.  These are detailed in the discussion of each of these sub-models in the NIST Technical Reference Guide \cite{CFAST_Tech_Guide_6}.